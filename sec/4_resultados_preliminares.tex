\section{Metodologia e Resultados Preliminares}
\subsection*{}

\begin{frame}{Metodologia}
    \begin{itemize}
        \item Avaliar:
        \begin{itemize}
            \item Número de \textbf{\textit{Cache Misses}}
            \item Tempo de execução
        \end{itemize}
        \item 3 algoritmos de Physical Design:
        \begin{itemize}
            \item Problema A: Verificar limites do chip
            \item Problema B: Estimar o tamanho das interconexões
            \item Problema C: Clusterização de elementos
        \end{itemize}
        \item Implementações:
        \begin{itemize}
            \item Abordagem tradicional: Object-Oriented Design (OOD)
            \item Abordagem proposta: Data-Oriented Design (DOD) 
        \end{itemize}
    \end{itemize}
\end{frame}

\begin{frame}{Infraestrutura}
     \begin{columns}
        \column{.6\linewidth}
        
            \begin{itemize}
                % \itemsep15pt 
                \item Ophidian
                \item ICCAD 2015 CAD Contest: 
                \begin{itemize}
                    \item 8 circuitos de tamanhos de \\ \textbf{768m} a \textbf{1.93M} de células
                \end{itemize}
                \item Linux workstation com processador Intel\textsuperscript{\textregistered} Core\textsuperscript{\textregistered} i5-4460 @ 3.20~GHz
                \begin{itemize}
                    \item 32GB~RAM (DDR3 @ 1600MHz)
                    \item Memória Cache:
                        \begin{itemize}
                            \item L1: 64KB
                            \item L2: 256KB
                            \item L3: 6144KB
                        \end{itemize}
                \end{itemize}
                \item 30 repetições para garantir um pequeno intervalo de confiança
                % \item Código fonte e experimentos disponíveis na Ophidian %\footnote{Disponível em: https://github.com/eclufsc/ophidian}
                % \item \textbf{Scenario ~A}: check if all circuit cells' positions lie within the circuit's boundaries
                % In this scenario only one property is necessary for each entity (cell position), so it fully explores the data locality provided by \ac{dod}.
                % \item \textbf{Scenario ~B}: computing the interconnection wirelength for all circuit nets
                % This scenario accesses different properties of different entities. Therefore, it cannot efficiently explore the data locality provided by \ac{dod}, since the properties of the pins in a single net may not be contiguous in memory, unless the property array is previously sorted.
            \end{itemize}
            
        \column{.4\linewidth}
            \pgfdeclareimage[width=\linewidth]{mr1}{img/results/architectureMemoryZeus.pdf}
            \begin{center}
                \pgfuseimage<1>{mr1}
            \end{center}
    \end{columns}
\end{frame}

\begin{frame}{Problema A: Verificar limites do chip}
    
    \pgfdeclareimage[width=0.3\linewidth]{problema}{img/tecnica/problemaA.pdf}
    \pgfdeclareimage[width=0.27\linewidth]{ood}{img/tecnica/classHierarchyChipBoundariesOOD.pdf}
    \pgfdeclareimage[width=0.5\linewidth]{dod}{img/tecnica/chipBoundariesDOD.pdf}

    \begin{minipage}[c][.3\textheight][c]{1\textwidth}
        \centering
        \vspace{1cm}
        \pgfuseimage<1>{problema}
    \end{minipage}
    \begin{minipage}[c][.7\textheight][c]{1\textwidth}
        \begin{columns}
            \column{.5\linewidth}
                \vspace{1cm}
                \begin{itemize}
                    \item Object-Oriented Design
                \end{itemize}
                \centering
                \vspace{.5cm}
                \pgfuseimage<1>{ood}
                \vspace{1.2cm}
            \column{.5\linewidth}
                \vspace{-2.5cm}
                \begin{itemize}
                    \item Data-Oriented Design
                \end{itemize}
                \vspace{0.5cm}
                \pgfuseimage<1>{dod}
        \end{columns}
    \end{minipage}
    
\end{frame}

\begin{frame}{Problema A: Verificar limites do chip}

    \pgfdeclareimage[width=0.5\linewidth]{miss_1}{img/results/miss_problem_A_1.pdf}
    \pgfdeclareimage[width=0.5\linewidth]{miss_2}{img/results/miss_problem_A_2.pdf}
    \pgfdeclareimage[width=0.5\linewidth]{runtime_1}{img/results/runtime_problem_A_1.pdf}
    \pgfdeclareimage[width=0.5\linewidth]{runtime_2}{img/results/runtime_problem_A_2.pdf}

    \begin{columns}
        \column{.5\linewidth}
            \only<1>{
                \centering
                Cache Misses
                
                \vspace{0.5cm}
                \pgfuseimage{miss_1}
            }
            \only<2-4>{
                \centering
                Cache Misses
                
                \vspace{0.5cm}
                \pgfuseimage{miss_2}
            }
        \column{.5\linewidth}
            \only<3>{
                \centering
                Tempo de execução
                
                \vspace{0.5cm}
                \pgfuseimage{runtime_1}
            }
            \only<4>{
                \centering
                Tempo de execução
                
                \vspace{0.5cm}
                \pgfuseimage{runtime_2}
            }
    \end{columns}
\end{frame}

\begin{frame}{Problema B: Estimar o tamanho das interconexões}

    \pgfdeclareimage[width=0.15\linewidth]{problema}{img/tecnica/problemaB.pdf}
    \pgfdeclareimage[width=0.3\linewidth]{ood}{img/tecnica/classHierarchyOOD.pdf}
    \pgfdeclareimage[width=0.5\linewidth]{dod}{img/tecnica/estimativa_interconection_dod.pdf}

    \begin{minipage}[c][.3\textheight][c]{1\textwidth}
        \centering
        \vspace{1.4cm}
        \pgfuseimage<1>{problema}
    \end{minipage}
    \begin{minipage}[c][.7\textheight][c]{1\textwidth}
        \begin{columns}
            \column{.5\linewidth}
                \begin{itemize}
                    \item Object-Oriented Design
                \end{itemize}
                \centering
                % \vspace{.5cm}
                \pgfuseimage<1>{ood}
                % \vspace{1.2cm}
            \column{.5\linewidth}
                \begin{itemize}
                    \item Data-Oriented Design
                \end{itemize}
                \pgfuseimage<1>{dod}
                \vspace{.5cm}
        \end{columns}
    \end{minipage}
        
\end{frame}

\begin{frame}{Problema B: Estimar o tamanho das interconexões}

    \pgfdeclareimage[width=0.5\linewidth]{miss_1}{img/results/miss_problem_B_1.pdf}
    \pgfdeclareimage[width=0.5\linewidth]{miss_2}{img/results/miss_problem_B_2.pdf}
    \pgfdeclareimage[width=0.5\linewidth]{runtime_1}{img/results/runtime_problem_B_1.pdf}
    \pgfdeclareimage[width=0.5\linewidth]{runtime_2}{img/results/runtime_problem_B_2.pdf}

    \begin{columns}
        \column{.5\linewidth}
            \only<1>{
                \centering
                Cache Misses
                
                \vspace{0.5cm}
                \pgfuseimage{miss_1}
            }
            \only<2-4>{
                \centering
                Cache Misses
                
                \vspace{0.5cm}
                \pgfuseimage{miss_2}
            }
        \column{.5\linewidth}
            \only<3>{
                \centering
                Tempo de execução
                
                \vspace{0.5cm}
                \pgfuseimage{runtime_1}
            }
            \only<4>{
                \centering
                Tempo de execução
                
                \vspace{0.5cm}
                \pgfuseimage{runtime_2}
            }
    \end{columns}
\end{frame}

\begin{frame}{Problema B: Estimar o tamanho das interconexões}

    \pgfdeclareimage[width=0.75\linewidth]{future}{img/results/future_work_problem_B.pdf}

    \begin{itemize}
        \item \textbf{Possível otimização}:
        \begin{itemize}
            \item Agrupar a posições dos pinos
        \end{itemize}
    \end{itemize}
    
    \centering
    \pgfuseimage<1>{future}
    
\end{frame}


\begin{frame}{Exemplo HPWL com agrupamento dos pinos}
    \pgfdeclareimage[width=0.8\linewidth]{hpwl0}{img/results/hpwl/hpwl_anim_0.pdf}
    \pgfdeclareimage[width=0.8\linewidth]{hpwl1}{img/results/hpwl/hpwl_anim_1.pdf}
    \pgfdeclareimage[width=0.8\linewidth]{hpwl2}{img/results/hpwl/hpwl_anim_2.pdf}
    \pgfdeclareimage[width=0.8\linewidth]{hpwl3}{img/results/hpwl/hpwl_anim_3.pdf}
    \pgfdeclareimage[width=0.8\linewidth]{hpwl4}{img/results/hpwl/hpwl_anim_4.pdf}
    \pgfdeclareimage[width=0.8\linewidth]{hpwl5}{img/results/hpwl/hpwl_anim_5.pdf}
    \centering
    \vspace{0.5cm}

    \pgfuseimage<1>{hpwl0}
    \pgfuseimage<2>{hpwl1}
    \pgfuseimage<3>{hpwl2}
    \pgfuseimage<4>{hpwl3}
    \pgfuseimage<5>{hpwl4}
    \pgfuseimage<6>{hpwl5}
    
    \begin{tikzpicture}[remember picture,overlay]
        % \node[rectangle, draw, minimum width = 2.cm, minimum height = 1.cm, color=red, rounded corners, very thick](window)at(1.45,2.55){};
        \node (window)at(5.5,6.5){\color{red}{\textbf{Ordem original}}};
        \node (window1)at(5.5,2.9){\color{green}{\textbf{Agrupando pinos}}};
        \node (window2)at(5.5,2.5){\color{green}{\textbf{de uma mesma net}}};
    \end{tikzpicture}
\end{frame}

\begin{frame}{Problema C: Clusterização de elementos}

    \pgfdeclareimage[width=0.2\linewidth]{problema}{img/tecnica/problemaC.pdf}
    \pgfdeclareimage[width=0.3\linewidth]{ood}{img/tecnica/registerClusterclassOOD.pdf}
    \pgfdeclareimage[width=0.5\linewidth]{dod}{img/tecnica/propertiesDOD.pdf}

    % \begin{minipage}[c][.3\textheight][c]{1\textwidth}
    %     \centering
    %     \vspace{1.4cm}
    %     \pgfuseimage<1>{problema}
    % \end{minipage}
    % \begin{minipage}[c][.7\textheight][c]{1\textwidth}
        \begin{columns}
            \column{.5\linewidth}
                \centering
                % \vspace{1.4cm}
                \pgfuseimage<1>{problema}
                \vspace{.5cm}
                \begin{itemize}
                    \item Object-Oriented Design
                \end{itemize}
                % \centering
                % \vspace{.5cm}
                \pgfuseimage<1>{ood}
                % \vspace{1.2cm}
            \column{.5\linewidth}
                \begin{itemize}
                    \item Data-Oriented Design
                \end{itemize}
                \pgfuseimage<1>{dod}
        \end{columns}
    % \end{minipage}
   
    % \begin{columns}
    %     \column{.5\linewidth}
    %         \begin{itemize}
    %             \item Object-Oriented Design
    %         \end{itemize}
    %         \pgfuseimage<1>{ood}
    %     \column{.5\linewidth}
    %         \begin{itemize}
    %             \item Data-Oriented Design
    %         \end{itemize}
    %         \pgfuseimage<1>{dod}
    % \end{columns}

\end{frame}

\begin{frame}{Problema C: Clusterização de elementos }

    \pgfdeclareimage[width=0.5\linewidth]{miss_1}{img/results/miss_problem_C_1_sequential.pdf}
    \pgfdeclareimage[width=0.5\linewidth]{miss_2}{img/results/miss_problem_C_2_sequential.pdf}
    \pgfdeclareimage[width=0.5\linewidth]{runtime_1}{img/results/runtime_problem_C_1_sequential.pdf}
    \pgfdeclareimage[width=0.5\linewidth]{runtime_2}{img/results/runtime_problem_C_2_sequential.pdf}

    \begin{itemize}
        \item Execução \textbf{Sequencial}
    \end{itemize}

    \vspace{0.3cm}

    \begin{columns}
        \column{.5\linewidth}
            \only<1>{
                \centering
                Cache Misses
                
                \vspace{0.5cm}
                \pgfuseimage{miss_1}
            }
            \only<2-4>{
                \centering
                Cache Misses
                
                \vspace{0.5cm}
                \pgfuseimage{miss_2}
            }
        \column{.5\linewidth}
            \only<3>{
                \centering
                Tempo de execução
                
                \vspace{0.5cm}
                \pgfuseimage{runtime_1}
            }
            \only<4>{
                \centering
                Tempo de execução
                
                \vspace{0.5cm}
                \pgfuseimage{runtime_2}
            }
    \end{columns}
\end{frame}

\begin{frame}{Problema C: Clusterização de elementos }

    \pgfdeclareimage[width=0.5\linewidth]{miss_1}{img/results/miss_problem_C_1_parallel.pdf}
    \pgfdeclareimage[width=0.5\linewidth]{miss_2}{img/results/miss_problem_C_2_parallel.pdf}
    \pgfdeclareimage[width=0.5\linewidth]{runtime_1}{img/results/runtime_problem_C_1_parallel.pdf}
    \pgfdeclareimage[width=0.5\linewidth]{runtime_2}{img/results/runtime_problem_C_2_parallel.pdf}

    \begin{itemize}
        \item Execução \textbf{Paralela}
    \end{itemize}

    \vspace{0.3cm}

    \begin{columns}
        \column{.5\linewidth}
            \only<1>{
                \centering
                Cache Misses
                
                \vspace{0.5cm}
                \pgfuseimage{miss_1}
            }
            \only<2-4>{
                \centering
                Cache Misses
                
                \vspace{0.5cm}
                \pgfuseimage{miss_2}
            }
        \column{.5\linewidth}
            \only<3>{
                \centering
                Tempo de execução
                
                \vspace{0.5cm}
                \pgfuseimage{runtime_1}
            }
            \only<4>{
                \centering
                Tempo de execução
                
                \vspace{0.5cm}
                \pgfuseimage{runtime_2}
            }
    \end{columns}
\end{frame}

